\section{Criticality Testing}

In this section we describe algorithms used to determine list-criticality of graphs. Recall that we are not storing the explicit list assignment $L$ for our graphs, so what we want to check is whether there exists a $L$ so that the graph is critical with respect to that $L$. However, even if $L$ was fixed, it would still be a computationally hard problem to determine criticality. What we will do instead is check for weaker properties, and therefore admit some false positives, that is, some graphs we identify as list-critical for which actually no suitable $L$ exist. Our hope is that the tests will be exhaustive enough so that finiteness results such as \ref{linearboundcycletheorem} still hold for the weaker properties we are testing, and our algorithms terminate. We will see that indeed, the algorithms described here work very well in practice at discarding non-critical graphs. 

\todo{explain that we work with graphs with prescribed list sizes}

\subsection{Degree properties}

We can start with an easy observation:

\begin{Observation}
In a $L$-critical graph, $d(v) \geq |L(v)|$ for all vertices $v$.
\end{Observation}

So if we find a vertex with degree less than the prescribed list size, we can conclude that the graph is not list-critical. 
However, this is a very weak test. We can incorporate another test concerning the vertices with $d(v) = |L(v)|$: there is
the following result by Gallai showing that the subgraph induced by those vertices must have a certain structure, generalizing 
the classical Brooks theorem for vertex coloring:

\begin{theorem}[Gallai \cite{gallaikritische}]
Let $G$ be a $L$-critical graph and let $H$ be the subgraph
of $H$ induced by the vertices with $d(v) = |L(v)|$. 
Then each $2$-connected component of $H$ is a complete graph or an odd cycle.
\end{theorem}

\subsection{Reducible Configurations}

\begin{figure}
\begin{tikzpicture}
	\VertexIII[y=1]{A}
	\VertexII[x=-2]{B}
	\VertexII[y=-1]{C}
	\VertexII[x=2]{D}
	
	\Edge(A)(B)
	\Edge(A)(C)
	\Edge(A)(D)
	\Edge(B)(C)
	\Edge(C)(D)
\end{tikzpicture}
\begin{tikzpicture}
	\VertexIV[y=1]{X}
	\VertexII[x=-3, y=-1]{A}
	\VertexII[x=-1, y=-1]{B}
	\VertexII[x=1, y=-1]{C}
	\VertexII[x=3, y=-1]{D}
	
	\Edge(X)(A)
	\Edge(X)(B)
	\Edge(X)(C)
	\Edge(X)(D)
	\Edge(A)(B)
	\Edge(B)(C)
	\Edge(C)(D)
\end{tikzpicture}
\end{figure}

\subsection{The Alon-Tarsi Method}
\subsection{Recursive Colorability Testing}

\begin{algorithm}[H]
\caption{Recursive Colorability Testing.}
\SetAlgoLined
\SetKwProg{Fn}{function}{}{end}
\Fn{containsColorableSubgraph(G)}{
\If{$G$ is empty} {
	return false\;
}
\If{alonTarsi($G$)} {
	return true\;
}
$H \gets \text{minimalNonColorable}(G)$\;
return containsColorableSubgraph($G \setminus H$)\;
}
\Fn{minimalNonColorable(G)}{
	\For{$v \in V(G)$} {
		\If{not alonTarsi(removeVertex($G$,$v$))} {
			return minimalNonColorable(removeVertex($G$,$v$))\;
		}
	}
	return $G$\;
}


\end{algorithm}
\subsection{Criticality Verification}
