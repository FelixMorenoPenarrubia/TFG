\chapter{Results and Further Study}

In this chapter we describe the computational results we got by executing our implementation
of the techniques described in the previous chapters, and briefly discuss next steps for
working on the 6-list-critical graphs on the torus problem.

\section{Computational Results}
\subsection{Generation of Cycle-Canvases}

\begin{table}[h]
\label{tab:cyclecanvases}
\centering
\begin{tabular}{l | l || l | l || l | l}
$\ell$ & \# & $\ell$ & \# & $\ell$ & \# \\
\hline
3 & 1 & 7  & 18    & 11 & 131221\\ 
4 & 1 & 8  & 145   & 12 & 1447449 \\
5 & 2 & 9  & 1260  & 13 & 16506284 \\ 
6 & 5 & 10 & 12518 & 14 & -- 
\end{tabular}
\caption{Number of critical cycle-canvases by cycle size}
\end{table}

In Table \ref{tab:cyclecanvases} we can see the number of cycle-canvases generated by our program
for sizes from $3$ to $13$. 
We see an exponential growth in the number of critical cycle-canvases, which is consistent with the
linear number of vertices bound of Theorem \ref{linearboundcycletheorem}. 
The base of the exponent appears to be approximately $10$.

Note that the number corresponds to \emph{candidates} for critical cycle-canvases (also including
the empty cycle-canvases, which are considered critical by our program for implementation ease), 
and the number of actual cycle-canvases could be slightly lower, especially in the higher cycle 
sizes. (This applies for all the tables in this chapter).

The generation of the critical cycle-canvases of cycle size $13$ takes around 10 hours in our 
computer. Our program uses too much memory for cycle-canvases of size $14$, because of the need to
store critical canvases in the queue and in the associative container used to eliminate duplicates.
We see the following possible improvements to ameliorate this:

\begin{itemize}
	\item Process the cycle-canvases in a depth-first rather than breadth-first way: that is,
	use a stack rather than a queue. Because the search space of critical cycle-canvases is 
	shallow with respect to the operation of adding a tripod, this will use less memory.
	\item Do not store the found cycle-canvases, just print them. This brings up the problem
	of how to handle duplicates: one possible way is to try to define a ``canonical'' sequence 		of tripod operations to generate any possible cycle-canvas, and only generate canvases
	through that canonical sequence of operations. This requires analyzing the symmetries of the
	graphs.
	\item Alternatively, compress the information of the generated graphs as much as possible
	while still retaining the ability to detect duplicates. Applying a compression hash function
	to the DFS transcript is the simplest example.

\end{itemize}

Also, we can also improve the time performance of the program via parallelization, since the program does criticality testing different graphs independently, and therefore can easily be 
parallelized.

We did in fact implement some of the proposals above, but it was not enough to achieve the 
generation of all critical cycle-canvases of cycle size $14$ in our computer. Nevertheless,
we believe that generating cycle-canvases of cycle size $14$ is feasible and can be achieved
by carefully optimizing the time and memory performance of our program or by running the program
in a computer with greater resources. However, the exponential growth in the number of critical 
cycle-canvases suggests that the practical limit might be $14$ or $15$. In particular, it is not 
feasible to generate all $~10^{10}$ critical cycle-canvases of cycle size $16$, which means that
if the appropiate constant in the Two Precolored Triangles Theorem is $5$, as we conjecture, we
will not be able to verify it by canvas strangulation.




\subsection{Generation of Prism-Canvases via Strangulation}



\begin{table}[h]
\label{tab:prismcanvases}
\centering
\begin{tabular}{l | l}
$d$ & \# \\
\hline
1 & 352 \\
2 & 1573\\ 
3 & 125 \\
\end{tabular}
\caption{Number of critical prism-canvases by distance between triangles}
\end{table}

Table \ref{tab:prismcanvases} shows the number of prism-canvases obtained by canvas strangulation
of cycle-canvases of cycle sizes $8$, $10$, $12$. The stark decline with prism-canvases at distance
$3$ gives us hopes that the number will reach $0$ by distance $5$ (or even $4$). 

Processing the critical cycle-canvases of cycle size $12$ to get the prism-canvases at distance $3$
takes about $30$ minutes in our computer, approximately the same time the cycle-canvas search 
program takes to generate the canvases of size $12$. The same optimizations, in particular
parallelization, can be applied to this program to make it faster. 

\subsection{Forbidden 3-3 Setting Approaches}

\begin{table}[h]
\label{tab:wedges}
\centering
\begin{tabular}{l | l}
$\ell$ & \# \\
\hline
1 & 1 \\
2 & 3 \\ 
3 & 22 \\
4 & 245 \\
5 & 3198 \\
6 & 44945 \\
\end{tabular}
\caption{Number of critical 3-3-forbidden wedges by path length}
\end{table}

\begin{table}[h]
\label{tab:biwedges}
\centering
\begin{tabular}{l | l | l}
$d$ & \# 2 Paths & \# Path \& Vertex \\
\hline
1 & 22   & 1\\
2 & 212  & 10\\ 
3 & 116  & 4 \\
4 & 14   & 0
\end{tabular}
\caption{Number of critical 3-3-forbidden biwedges by distance between precolored paths}
\end{table}

Table \ref{tab:wedges} shows the number of critical 3-3 forbidden wedges by path length up to
length $6$ and Table \ref{tab:biwedges} shows the number of biwedges (both those with two precolored
paths of length one and those with one path of length one and one path of length zero) generated
from those wedges by the distance between precolored paths up to distance $4$. We computationally performed the proof steps explained in Section . and therefore proved Conjecture ., which
we now restate here as two theorems. 

\todo{refs section and conjecture}

\begin{theorem}
There exist only finitely many 3-3-forbidden critical biwedges.
\end{theorem}

\begin{theorem}
Let $G$ be a plane graph with outer walk $C$, let $P_1, P_2$ be two paths of length $1$ in $C$, and let
$L$ be a list assignment for $G$. If all the following conditions are satisfied:

\begin{enumerate}
	\item $|L(v)| \geq 5 \, \forall v \in V(G) \setminus V(C)$.
	\item $|L(v)| \geq 3 \, \forall v \in V(C) \setminus (V(P_1) \cup V(P_2))$
	\item $P_1$, $P_2$ are $L$-colorable.
	\item There does not exist $uv \in E(G)$ with $|L(u)| = |L(v)| = 3$.
	\item $d(P_1, P_2) \geq 5$.
\end{enumerate}

then $G$ is $L$-colorable. 
\end{theorem}




\subsection{Criticality Strength Approaches}

\begin{table}[h]
\label{tab:strongwedges}
\centering
\begin{tabular}{l | l}
$\ell$ & \# \\
\hline
2 & 4 \\
3 & 269\\ 
4 & 31370 \\
\end{tabular}
\caption{Number of strong critical wedges by path length}
\end{table}

We can see in Table \cite{tab:strongwedges} that the number of strong critical wedges in the usual
canvas setting grows very fast, possibly exponentially with a base of around $100$. Therefore,
it is not feasible to reach wedges with length $9$, which are the ones we would need in the described 
approach. We tried other approaches using the
criticality strength ideas, but they suffer from similar combinatorial explosion issues. 

\section{Conclusions and Further Study}

We hope that the work done in this thesis can serve as a first step in the quest for finding the
set of 6-list-critical graphs on the torus. Based on the results we have gotten above and what 
we have studied about the problem, we outline some possible next steps to be taken:

\begin{itemize}
	\item Implement the corresponding optimizations to the critical cycle-canvas search and
	canvas strangulation programs in order to obtain the list of critical cycle-canvases with
	cycle size $14$ and the list of critical prism-canvases with triangle distance $4$. 
	We discussed the details of what those optimizations could be in the previous 
	section.
	\item Try to prove the Two Precolored Triangles Theorem with the tightest bound. 
	This is the main obstacle we have faced for our plan. 
	Our approaches have not been successful: the 3-3-forbidden
	setting worked well for the biwedges but it is not conceptually clear how to extend
	it to the precolored triangle setting, and the approaches based on generating only strong
	critical graphs have computational issues because there are too many graphs. Perhaps
	more work and more complex strategies can make these approaches successful, or
	perhaps new ideas are needed here.
	\item Implement gluing of prism-canvases along the precolored triangles to generate
	new candidates for critical prism-canvases with a separating triangle. This may allow
	us to get some critical prism-canvases with higher distances (if those exist) without
	having to strangulate cycle-canvases of large cycle size.
	\item Implement resconstruction of critical $6$-list-critical graphs on the torus
	from critical cycle-canvases and prism-canvases. This will allow us to see if we have
	found any counterexample to Conjecture \ref{torusconjecture}, and will at least allow
	us to prove a weaker result about the colorability of graphs on the torus with bounded
	distance between non-contractible triangles even if we do not prove the Two Precolored
	Triangles Theorem with a good bound. 
	\item Think about what to do for graphs on the torus without non-contractible triangles.
	
\end{itemize}






