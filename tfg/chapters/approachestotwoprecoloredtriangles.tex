\chapter{Approaches to the Two Precolored Triangles Theorem}

In this chapter we explain different approaches to proving Conjecture \ref{twotriangleconjecture}.

\section{Cycle-Canvas Strangulation}

We can generate all critical prism-canvases whose two triangles are at distance $d$ from critical
cycle-canvases with cycle size $2d+6$. Here is how:


\begin{proposition}
Let $G$ be a plane graph $(T_1 \cup T_2)$-critical with respect to some list assignment $L$, 
where $T_1$ and $T_2$ are two triangles 
and $P$ is a shortest path between them of length $d$. Let $G'$ be the graph
obtained by ``duplicating'' the $d+1$ vertices of path $P$, so that the edges of the path $P$ are
duplicated and the other neighbors of the duplicated vertices are now neighbors of the vertex
corresponding to the side of the path in which the neighbors were in (see Figure 
\ref{fig:strangulation}). Let $C$ be the corresponding cycle of length $2d+6$ that is newly formed
with the duplicated vertices and the vertices of the triangles. Then $G'$ is $C$-critical (with 
respect to the naturally corresponding list assignment $L'$).
\end{proposition}
\begin{proof}
By the Extension Lemma (\ref{extensionlemma}), $G$ is $(T_1 \cup T_2 \cup P)$-critical.
Now, the result follows by the Duplication Lemma (\ref{duplicationlemma}).
\end{proof}

\begin{figure}
\label{fig:strangulation}
\missingfigure{canvas strangulation}
\caption{Illustration of Canvas Strangulation}
\end{figure}

Now, we use this result in the backwards direction (as in Figure \ref{fig:strangulation}):
from critical cycle-canvases of cycle size $2d+6$, we generate candidates for critical 
prism-canvases by identifying $d+1$ consecutive vertices in the precolored cycle with the opposite
segment of $d+1$ consecutive vertices in the precolored cycle (it is useful to visualize 
the``inverted'' canvas, so that the precolored cycle is no longer the outer face but a cycle
bounding an interior face). Note that from each cycle-canvas there are $d+3$ ways to identify
the vertices. After criticality testing all candidates from all cycle-canvases of size $2d+6$, we
obtain all critical prism-canvases of distance $d$.

This can serve to get the list of
all critical prism-canvases with the triangles at a certain distance, which is useful
for a part of our plan and to experimentally determine the value of the right distance constant
in Conjecture \ref{twotriangleconjecture}.
However it is not useful, just by itself, to prove that there are no such critical prism-canvases
for large enough distances. 

\section{The Forbidden 3-3 Setting}

Recall Postle's approach: Postle, like Thomassen, works in a setting where the vertices of the 
outer face of the graphs are allowed to have lists of size $3$. 
For proving the Two Precolored Triangles Theorem, this is useful because a shortest path
between the triangles can be precolored to obtain a graph with two precolored paths
of length $1$ and vertices with list size at least $3$ in the outer face. The problem
is that there are infinitely many critical graphs in such a setting. The question is: can
we find some restriction on the list sizes so that:

\begin{enumerate}
	\item It is restrictive enough so that there are only finitely many $(P_1 \cup P_2)$-critical
	graphs, where $P_1, P_2$ are paths of length $1$ in the outer face, but
	\item It is not too restrictive, so it is useful and allows reductions like in the proof of
	Thomassen's Theorem.
\end{enumerate}

The setting we propose is: require vertices in the outer face to have list size at least $3$, and
also additionally require that no two adjacent non-precolored vertices have list size less than $4$. 
Note that in this setting, we cannot have the infinte amount of bellows that prevented
path-canvases with path length $2$ from being colorable, or the coloring harmonicas that prevented
canvases with a path of length $1$ and a path of length $0$ from being colorable.

In fact, this setting was first introduced in \cite{crossingsfarapart} to prove that there
are only finitely many $P$-critical graphs in this setting,
where $P$ is a path of length $2$, and as we will
see there are also only finitely many critical graphs for longer path lengths and for two 
paths of length $1$.

\begin{definition}
	We say that $(G, S, L)$ is a \emph{3-3-forbidden canvas} if $G$ is a connected plane graph
 with outer walk $C$, $S$ is a subgraph of $C$, and $L$ is a list assignment
  such that $|L(v)| \geq 5 \, \forall v \in V(G) \setminus V(C)$,
   $|L(v)| \geq 3 \, \forall v \in V(C) \setminus V(S)$ and for all $uv \in E(G)$ with 
   $u, v \not\in V(S)$ we have that $\max\{|L(u)|, |L(v)|\} \geq 4$. If $S$ is a path,
    we say $(G, S, L)$ is a \emph{3-3-forbidden wedge} (or just \emph{wedge} in this section). 
    If $S = P_1 \cup P_2$ with $P_1$, $P_2$ paths of length at most $1$, we say that $(G, S, L)$ is a
    \emph{3-3-forbidden biwedge}. We say that $(G, S, L)$ is critical if $G$ is $S$-critical
    with respect to $L$.
\end{definition}

Since in our program we do not store the list assignments for the graphs but we do store the 
prescribed list sizes for each vertex (that is, whether a vertex can have a list size of $3$ or
is forced to have $\geq 4$ colors), we will sometimes abuse terminology
 use the terms critical wedge and critical biwedge
to refer to those graphs with prescribed list sizes but not with a fixed list assignment.

\subsection{The Forbidden 3-3 Reduction}

Before we state the reduction that we can use in this setting, we need the following result:

\begin{proposition}
\ref{twocriticalwedgesprop}
There are only two critical 3-3-forbidden wedges with path length $2$:

\begin{enumerate}
	\item The three-vertex graph consisting of $P = p_0p_1p_2$ with an additional edge joining
	$p_0$ and $p_2$.
	\item The four vertex graph consisting of $P$ plus one vertex with list of size three joined
	to all three vertices of $P$.
\end{enumerate}

\end{proposition}

This is proven in \cite{crossingsfarapart} using the reduction we are about to describe,
but note that we can also prove it by generating critical wedges using the algorithm in 
Section \ref{sec:generationwedges} and seeing that it halts after generating only these two graphs.

\begin{definition}
We say that a 3-3-forbidden canvas $(G, S, L)$ is \emph{reducible} if the following holds:
\begin{enumerate}
	\item It is $2$-connected.
	\item The outer cycle has no chords.
	\item There is a precolored vertex $p_0 \in S$ so that the next four vertices 
	$v_1$, $v_2$, $v_3$, $v_4$ in clockwise order in the outer cycle are not in $S$. 
\end{enumerate}
\end{definition}

Let $(G, S, L)$ be a critical reducible 3-3-forbidden canvas, and consider the list sizes of
the vertices $v_1$, $v_2$, $v_3$, $v_4$ as in the definition of reducible. We are going to consider
different cases according to those list sizes, and in each case we will select a subset $X$ of those
vertices and a $L$-coloring on some of the vertices in $X$. 

\begin{figure}
\label{fig:reductioncases}
\centering
\missingfigure{reduction cases}
\caption{Illustration for the cases in the reduction}
\end{figure}

\begin{itemize}
	\item [X1] If $|L(v_2)| \geq 4$ and $|L(v_3)| \geq 4$, then $X = \{v_1\}$ and 
	$\phi(v_1) \in L(v_1) \setminus L(p_0)$ is chosen arbitrarily.
	\item [X2] If $|L(v_2)| \geq 4$ and $|L(v_3)| = 3$, then $X = \{v_1, v_2\}$ and $\phi$ is 
	chosen so that $\phi(v_2) \in L(v_2) \setminus L(v_3)$ and 
	$\phi(v_1) \in L(v_1) \setminus (L(p_0) \cup \{\phi(v_2)\})$.
	\item [X3] If $|L(v_2)| = 3$, and either $|L(v_4)| \neq 3$ or $|L(v_3)| \geq 5$, then 
	$X = \{v_2\}$ and $\phi(v_2) \in L(v_2)$ is chosen arbitrarily. 
	\item [X4] If $|L(v_2)| = 3$, $|L(v_3)| = 4$ and $|L(v_4)| = 3$, then:
	\begin{itemize}
		\item [X4a] If $v_1$, $v_2$ and $v_3$ do not have a common neighbor or 
		$|L(v_1)| \geq 5$, then $X = \{v_2, v_3\}$ and $\phi$ is chosen so that
		$\phi(v_3) \in L(v_3) \setminus L(v_4)$ and 
		$\phi(v_2) \in L(v_2) \setminus \{\phi(v_3)\}$.
		\item [X4b] If $v_1$, $v_2$ and $v_3$ have a common neighbor and $|L(v_1)|$ = 4, then
		$X = \{v_1, v_2, v_3\}$ and $\phi$ is chosen so that 
		$\phi(v_3) \in L(v_3) \setminus L(v_4)$, 
		$\phi(v_1) \in L(v_1) \setminus L(p_0)$ 
		and either at least one of $\phi(v_1)$ and $\phi(v_3)$
		does not belong to $L(v_2)$ or $\phi(v_1) = \phi(v_3)$. 
		The vertex $v_2$ is left uncolored.
	\end{itemize}
\end{itemize}

Let $G' = G \setminus X$ and let $L'$ be the list assignment obtained from $L$ by removing the colors 
of the vertices of $X$ according to $\phi$ from the lists of their neighbors (if a vertex of $X$ is not
colored according to $\phi$, we do not remove any colors for it). By the choice of $X$ and $\phi$,
any $L'$-coloring of $G'$ can be extended to an $L$-coloring of $G$, thus $G'$ is not $L'$-colorable.

This can imply one of these two options:

\begin{enumerate}
	
	\item $(G', S, L')$ contains a critical 3-3-forbidden canvas as a subgraph (and one which is 
	smaller than $(G, S, L)$.
	\item $(G', S, L')$ is not a 3-3-forbidden canvas.
\end{enumerate}

The second option can only happen if $G'$ contains two adjacent vertices with list size $3$. 
Let us call those vertices $u, v$. Because $G$ had no chords and because Lemma \ref{gluinglemma} and
Proposition \ref{twocriticalwedgesprop} applied to vertices of $G$ 
with two neighbors in $X$, we can conclude that $u$ and $v$ form a chord of the outer walk of $G'$.


\subsection{Plan for Critical Biwedges}

We will apply the above reduction to prove the following:

\begin{conjecture}
\label{biwedgeconjecture}
Let $(G, P_1 \cup P_2, f)$ be a tuple such that $G$ is a plane graph, $P_1$, $P_2$ are two length one 
paths in the outer walk of $G$, and $f$ is a list size assignment as in the definition of 3-3-forbidden
canvases. If there exists an $f$-list-assignment $L$ such that $(G, P_1 \cup P_2, L)$ is a critical
3-3-forbidden biwedge, then $(G, P_1 \cup P_2, f) \in \mathcal{S}$, where $\mathcal{S}$ is an explicit
finite set (to be determined computationally) such that all the members of $\mathcal{S}$ satisfy
$d(P_1, P_2) \leq 4$.
\end{conjecture}

In order to prove this, we do the following:

We can generate all critical biwedges with distance between paths $d$ by fusing all pairs of critical
wedges with path lengths $d+1$ or path lengths $d$ and $d+2$ as in figure \ref{fig:fusingwedges}
and testing criticality. 
By Lemma \ref{extensionlemma} applied to the shortest path between the two paths and Lemma 
\ref{gluinglemma}, all critical biwedges can be decomposed in this way. 

\begin{figure}
\label{fig:fusingwedges}
\missingfigure{fusing wedges}
\caption{Fusing wedges along the shortest path of the biwedge.}
\end{figure}

We let $\mathcal{S}$ be the set of critical biwedges with distance between paths
up to $4$ obtained by that procedure.

To prove the conjecture, assume for a contradiction that there is a critical biwedge $(G, P_1 \cup P_2,
f) \not\in \mathcal{S}$, and choose such a counterexample with the smallest $V(G)$. 

Assume $G$ has a chord or cutvertex. Then, by Lemma \ref{gluinglemma}, $G$ can be decomposed into
two smaller biwedges fused along the chord or cutvertex. We should verify computationally the following
statement:

\begin{proposition}
If any $B_1, B_2 \in \mathcal{S}$ are fused along one of their precolored paths to create a biwedge
$(G, P_1 \cup P_2, f)$ with $d(P_1, P_2) \geq 5$, then $(G, P_1 \cup P_2, f)$ is not critical.
\end{proposition}

So we conclude that $G$ can not have a chord or cutvertex. 

We must have $d(P_1, P_2) \geq 5$, since all critical biwedges with distance up to $4$ are in $S$.
Now, note that our biwedge is reducible. We apply the reduction. If $G$ contains as a 
subgraph a smaller critical biwedge, then that critical biwedge is in $\mathcal{S}$ 
and therefore has distance
at most $4$ between precolored paths, so $G$ shoud have distance at most $4$ between paths, 
contradiction.

Therefore we have that the resulting graph $G'$ after performing the reduction has a chord $uv$ with
two vertices of list size $3$. And $G'$ is not $L'$-colorable, so there exists a coloring of 
$P_1 \cup P_2$ that does not extend to $G'$, so $G'$ contains a subgraph $G''$ which is 
$(P_1 \cup P_2)$-critical with respect to $L'$. Now, $G''$ is not a critical biwedge
because of the $uv$ chord, but it by Lemma \ref{gluinglemma} it can be decomposed into two
critical smaller critical biwedges along the chord. Therefore, we must verify computationally
the following statement:

\begin{proposition}
If any $B_1, B_2 \in \mathcal{S}$ are fused along one of their precolored paths to create a 
triple $(G, P_1 \cup P_2, f)$
with \textbf{one chord between vertices with prescribed list sizes $3$ but no other adjacencies
between vertices with prescribed list size $3$}, then either $(G, P_1 \cup P_2, f)$ is not critical
\textbf{or} $d(P_1, P_2) \leq 4$. 
\end{proposition}
  
In order to reach a contradiction and conclude that such a counterexample must not exist. 

\subsection{Approach for Critical Triangle-Wedges} 

We can not use Postle's technique of precoloring a shortest path between the two
triangles to conclude the result for two triangles from the result for biwedges, because the
requirement that no two vertices adjacent vertices have list of size $3$ is too strong for
that.

Instead, what we can try is to use the reduction again. We generalize the two precolored triangles
setting as follows:

\begin{definition}
We say that $(G, S = P \cup T, L)$ is a \emph{3-3-forbidden triangle-wedge} if $G$ is a connected plane 
graph
 with outer walk $C$, $P$ is a  path of length $1$ of $C$, $T$ is a triangle bounding a face
 of $G$ (not necessarily incident with $C$)
 , and $L$ is a list assignment
  such that $|L(v)| \geq 5 \, \forall v \in V(G) \setminus (V(C) \cup V(T))$,
   $|L(v)| \geq 3 \, \forall v \in V(C) \setminus V(S)$ and for all $uv \in E(G)$ with 
   $u, v \not\in V(S)$ we have that $\max\{|L(u)|, |L(v)|\} \geq 4$. 
\end{definition}

Note that the critical prism-canvases can be retrieved from the critical 3-3-forbidden 
triangle-wedges by selecting those whose outer face is a triangle with list size $3$ in the outer
face. 

We can try a similar strategy as with the biwedges in order to prove that all the critical 
triangle-wedges belong to a critical set: construct all candidates in which the triangle is at a 
small distance from the outer face, and then apply the reduction to show that there are no critical
graphs with a larger distance.

But the problem is, here constructing all the graphs for which the triangle is at a small distance
from the outer face is much more complicated than for the biwedges.
It also requires larger critical wedges as building blocks, which we are less likely to be able to
be computed in reasonable time. In theory, this approach could be carried out with access to large
computational resources, but it seems like something smarter is needed. 

\section{Criticality Strength}

\subsection{The One Precolored Triangle Setting}

\subsection{Definition of Criticality Strength}

\subsection{Strong Critical Wedges}
